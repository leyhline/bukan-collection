\documentclass{article}
\usepackage[utf8]{inputenc}
\usepackage[T1]{fontenc}

\usepackage[backend=biber,style=numeric]{biblatex}
\addbibresource{references.bib}

\title{Identifying Woodblocks from the Bukan Collection}
\author{Thomas Leyh}
\date{March 20th, 2020}

\begin{document}

\maketitle

\section{Introduction}

The \emph{Center for Open Data in the Humanities} (CODH) is a joint research institution in Tokyo, Japan. Its goal is the promotion of data-driven research in the humanities.\cite{kitamoto2017codh} For this purpose they were releasing a number of data sets, all of them related to Japanese history and arts. This work started out by specifically looking at the \emph{Bukan Collection}\cite{}, around 300 books with information about government officials during 1700 to 1900. So, how can we assist humanities researchers with techniques from Computer Science?

By applying well-known algorithms from Computer Vision on the books' pages, without using information about the written content, we were able to develop a reliable system for spotting and visualizing similarities. This might be a first step into building a timeline of a book's prints, thus exposing some events hidden in there.

But more importantly, this is an example of how even conservative algorithms can give compelling results by using a few reasonable assumptions on the corresponding data. Even basic computer-aided quantitative analysis might reveal information that is near invisible of a human researcher.

\subsection{The Bukan Collection}

\section{Method}

\printbibliography

\end{document}
